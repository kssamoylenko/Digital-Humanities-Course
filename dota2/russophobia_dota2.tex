\documentclass{beamer}

\mode<presentation>
{
  \usetheme{Madrid}      
  \usecolortheme{seahorse}
  \usefonttheme{serif} 
  \setbeamertemplate{navigation symbols}{}
  \setbeamertemplate{caption}[numbered]
} 

%%% Работа с русским языком
\usepackage{cmap}					% поиск в PDF
\usepackage[T2A]{fontenc}			% кодировка
\usepackage[utf8]{inputenc}			% кодировка исходного текста
\usepackage[english,portuguese, main=russian]{babel}	% локализация и переносы

%Диаграмы и рисунки
\usepackage{smartdiagram}
\usepackage{graphicx}
\graphicspath{{pictures/}}
\DeclareGraphicsExtensions{.pdf,.png,.jpg}
\usepackage{pifont}

\usepackage[english]{babel}
%\usepackage[utf8x]{inputenc}
%\usepackage{xcolor}
\usepackage{listings}
\lstset
{
    language=[LaTeX]TeX,
    breaklines=true,
    basicstyle=\tt\scriptsize,
    %commentstyle=\color{green}
    keywordstyle=\color{blue},
    %stringstyle=\color{black}
    identifierstyle=\color{magenta},
}

\title[Digital Humanities]{Русофобия в DotA2. Критический дискурсивный анализ онлайн-дискриминации}
\author{Анастасия Кузнецова}
\institute{НИУ ВШЭ}
\date{13 июня 2018}



\begin{document}
\begin{frame}
  \titlepage
\end{frame}


\begin{frame}{Что такое DotA2?}
		\begin{figure}
			\includegraphics[height=6.8cm]{dota.jpeg}
		\end{figure}
\end{frame}
\begin{frame}{Что такое DotA2?}
	\textbf{<<Defence of Ancients>>} \\ (MOBA or multiplayer online battle arena)
	\begin{itemize}
		\item Битва между двумя командами по 5 игроков. 
		\item Цель игры -- разрушить базу противника. 
		\item Каждый игрок может выбирать себе уникального героя с уникальными способностями.
	\end{itemize}
	\textbf{Статья:} \\
	\textit{Albin Wagener.} Russophobia in DotA 2. A critical discursive analysis of online discrimination, 2018. 
\end{frame}

\begin{frame}{Исследование}
\textbf{Дискурсивный анализ. Определение дискурса} \\
Очень глубокое погружение в анализ понятия. Автор статьи понимает под дискурсом:
\begin{itemize}
	\item Лингвистические (и не только) элементы семиотики, сосредоточенные на отношениях с элементами социального и материального. 
	\item Форму социальной практики. Он обусловлен взаимоотношениями между конкретным дискурсивным событием и ситуацией, социальными институтами, общественной структурой. 
	\item Область прагматики, изучающей условия использования семантических знаков.
	
\end{itemize}

\end{frame}

\begin{frame}{Корпус}
	\textbf{Типы дискурса}
	\begin{itemize}
		\item Общение в чате в процессе игры
		\item Общение на форумах вне игрового процесса. 
	\end{itemize}
	Воспроизведение стереотипов о России и русских в дискурсе. \\
	\textbf{Данные корпуса}
		\begin{itemize}
			\item Форумы и топики в форумах по ключевым словам <<Russians>>, <<DotA2>> (официальные форумы, администрируемые создателями игры, неофициальные форумы, созданные комьюнити).
			\item 1058 высказываний из 7 источников по 9 темам. 
		\end{itemize}
\end{frame}

\begin{frame}{Корпус}
	\begin{figure}
		\includegraphics[scale=0.3]{img7.png}
	\end{figure}
\end{frame}

\begin{frame}{Isolate Russian DOTA 2 players on their own servers}
	\begin{figure}
		\includegraphics[scale=0.5]{img8.png}
	\end{figure}
\end{frame}

\begin{frame}{Методы исследования}
	\textbf{Методы} 
	\begin{itemize}
		\item Анализ коллокаций
		\item Составление семантико-тематических облаков.
		\item Проксимизация (выделение отношений между сущностями в пространстве дискурса). 
		\item Пространственно-временная аксиологическая интерпретация.
	\end{itemize}
	\textbf{Инструменты}\\
	WordSmith (Oxford)\\
	Tropes
\end{frame}

\begin{frame}{Анализ данных}
	\textbf{Идея дейктических центров (P. Cap, 2013, 2014)}
	\begin{itemize}
		\item \textit{Outside Deictic Center} (ODC)
		\item  \textit{Inside Deictic Center} (IDC) 
	\end{itemize}
	Попытка выделить две полярных группы противопоставленных друг другу в дискурсе: русскоязычные пользователи vs. остальные игроки. \\
	Анализ частотности слов позволяет выделить две две группы действующих лиц по употребляемым ими словам: \\
	\begin{enumerate}
		\item 'I', 'me', 'we' (598 высказываний)
		\item 'They', 'Them', 'Their' (493 высказывания)
	\end{enumerate}
\end{frame}

\begin{frame}{Анализ данных}
	\begin{figure}
		\caption{Распределение высказываний по семантическим областям}
		\includegraphics[height=6.8cm]{img1.png}
	\end{figure}
\end{frame}

\begin{frame}{Анализ данных}
	\begin{figure}
		\caption{Таблица частотности по семантическим областям}
		\includegraphics[height=6.8cm]{img2.png}
	\end{figure}
\end{frame}

\begin{frame}{Анализ данных}
	\begin{figure}
		\caption{Частотные коллокации для слова <<сервер>> (server)}
		\includegraphics[height=6.8cm]{img3.png}
	\end{figure}
\end{frame}

\begin{frame}{Анализ данных}
	\begin{figure}
		\caption{Частотные коллокации для слова <<плохой>> (bad)}
		\includegraphics[scale=0.4]{img4.png}
	\end{figure}
\end{frame}

\begin{frame}{Анализ данных}
	\begin{figure}
		\caption{Inside Deictic Center}
		\includegraphics[scale=0.4]{img5.png}
	\end{figure}
\end{frame}

\begin{frame}{Анализ данных}
	\begin{figure}
		\caption{Outside Deictic Center}
		\includegraphics[scale=0.4]{img6.png}
	\end{figure}
\end{frame}


\begin{frame}{Результаты}
\begin{itemize}
	\item Русские игроки не демонстрируют должно знания английского языка, чем доставляют дискомфорт остальным;
	\item Необходимо выделить отдельный сервер для русскоговорящей аудитории (решение проблемы) ;
	\item Русскоговорящие игроки воспринимаются исключительно как жители России. Аудитория не имеет представления о разнообразии русскоязычного населения на территории Восточной Европы \ding{221} Воспроизводятся стандартные стереотипы о России.
	\item  При том, если в игре используются другие языки кроме русского, это не создает больших проблем. 
	\item Наличие в команде русскоязычного игрока автоматически ведет к поражению. 
	
\end{itemize}
\end{frame}

\end{document}
